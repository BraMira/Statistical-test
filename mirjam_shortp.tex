\documentclass{beamer}


\setbeamertemplate{background canvas}[vertical shading][bottom=white,top=structure.fg!25]
% or whatever

\usetheme{Warsaw}
\setbeamertemplate{headline}{}
\setbeamertemplate{footline}{}
\setbeamersize{text margin left=0.5cm}
  
\usepackage[english]{babel}
% or whatever

\usepackage[utf8]{inputenc}
% or whatever

\usepackage{times}
\usepackage[T1]{fontenc}
% Or whatever. Note that the encoding and the font should match. If T1
% does not look nice, try deleting the line with the fontenc.

\usepackage{amsfonts}
\usepackage{amsmath,amsthm}
\usepackage{url}

\title[Statistical test]
{Statistical test}
\author[Mirjam Pergar]
{\textbf{Student:}  Mirjam Pergar\\
\textbf{Proposer and mentor:} dr. asist. Gregor Šega
}
\institute[Fakuleta za matematiko in fiziko]

\date[April 20th 2017] % (optional)
{April 20th 2017}


\begin{document}

\begin{frame}{Speaker's Name}{About Our Next Speaker}

  \begin{itemize}
  \item
    Current affiliation of Speaker's Name

    % Examples:
    \begin{itemize}
    \item
      Professor of mathematics, University of Wherever.
    \item
      Junior partner at company X.
    \item
      Speaker for organization/project X.
    \end{itemize}
  \item
    Experience and achievements
    % Optional. Use this if it is appropriate to slightly flatter the
    % speaker, for example if the speaker has been invited. 
    % Using subitems, list things that make the speaker look
    % interesting and competent.

    % Examples:
    \begin{itemize}
    \item
      Academic degree, but only if appropriate
    \item
      Current and/or previous positions, possibly with dates
    \item
      Publications (possibly just number of publications)
    \item
      Awards, prizes
    \end{itemize}
  \item
    Concerning today's talk
    % Optional. Use this to point out specific experiences/knowledge
    % of the speaker that are important for the talk and that do not
    % follow from the above.

    % Examples:
    \begin{itemize}
    \item
      Expert who has worked in the field/project for X month/years.
    \item
      Will present his/her/group's/company's research on the subject.
    \item
      Will summarize project report or current project status.
    \end{itemize}
  \end{itemize}  
\end{frame}

\end{document}

library(matrixStats)


N <- c(seq(2,10),seq(12,20,2),25,30,40,60);
R <- c(seq(2,10),15,20);
alpha <- 0.01;
alpha_0.01 <- data.frame(row.names = N)

loops <- 10^6;
time0 <- proc.time() #we start the timer
for(r in R){
  for(n in N){
    rr <- seq(1,r);
    X <- matrix(rnorm(n*r*loops),nrow = r*loops,ncol=n);#generate a random matrix
    S <- rowVars(X); #row sample variances
    dim(S)<-c(loops,r);#vector -> matrix
    F <- sapply(1:loops,function(x) round(sum(sort(S[x,])*(2*rr -1) / r)/sum(S[x,]),3))
    # alpha_0.01['N','R'] <- quantile(F,1-alpha); #95% jih je levo od te meje
  }
}

time <- proc.time()-time0 #we stop the timer
print(time)
#pdf('Primer_1e+6_1percent.pdf')
# x<-seq(0,10,0.01);
# hist(F,freq = FALSE,xlim = c(min(F)-0.2,max(F)+0.2))
# curve(dnorm(x,mean=mean(F),sd=sd(F)),add=TRUE, col = 'red')
# abline(v=quantile(F,0.99),col='green',lwd=2)
# dev.off()

# user  system elapsed
# 14.95    0.04   15.06
